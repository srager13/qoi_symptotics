
\section{Introduction}
\label{sec:intro}

%area
Symptotic analysis is a relatively new framework for characterizing practical network scalability instead of using common asymptotic analysis.  Introduced in \cite{scalability_manets_theory_vs_practice}, symptotics is extremely useful for applications and network designers that are interested in determining the limits of a specific network implementation as well as how various factors affect these limits in terms of scalability.  For example, imagine designing an emergency ad hoc network to quickly replace destroyed infrastructure after a natural disaster or to exchange information without the use of government-controlled infrastructure in a time of political unrest.  Quantifying the traffic limitations, what topology allows for the largest feasible network size, or how load balancing will impact capacity are all crucial to successfully designing the most effective network.

%problem
%why not solved
Understanding the effects of Quality of Information (QoI) requirements on network scalability is also extremely important, because as seen in many recent fields of study like \cite{qoi_aware_tactical_mil_nets, qoi_max_two_hop_nets, opt_qoi_data_collection_bijarbooneh, qoi_aware_mobile_apps, explor_vs_exploit, qoi_sched_task_proc_nets}, the actual benefits of a network in terms of data utility may vary greatly from the standard metrics often used to describe Quality of Service.  As we will discuss later, timeliness of data collection is one such quality that is crucial in many applications, but is not captured without explicit consideration.  %Increasingly, network nodes are becoming more capable of affecting QoI through data fusion, compression, selection, etc.  
For this reason, we aim to begin understanding the connections between QoI requirements and practical network scalability in this work.

%insight
Specifically, we consider the practical effects of real networks, including protocol overhead, contention, and traffic loads, in conjunction with QoI to build a framework that provides upper limits on network sizes for defined QoI requirements.  Utilizing QoI in this framework is important because the relationship between throughput and QoI is often non-linear, a notion that will be supported in Section \ref{sec:qoi_model}.  We use timely, similarity-based image collection to provide motivation and concrete applications with which we test our methodology and provide results.  The model, though, is designed to be general so that any relationship between data rates and QoI can be inserted for scalability analysis.

%contribution
The results in Section \ref{sec:results} show several key insights.  First we show the maximum sizes to which the networks can scale for different requested levels of QoI. The QoI depends on timeliness and one of two metrics considered: completeness or diversity.  These attributes are defined by the sum similarity of collected images resulting from Top-K queries for completeness, and sum dissimilarity of images collected using the greedy spanner algorithm.  These definitions will be explained in more detail in Section \ref{sec:qoi_model}.  With these results, we show that the network scalability considerably reduces with higher completeness and diversity requests, as well as with stringent timeliness requirements.  

Finally, we identify the trade-off of a given QoI requirement resulting in both a minimum required network size to provide a required number of images and the maximum feasible network size able to support that amount of necessary traffic.  We identify the region of QoI requests where the former does not exceed the latter, and, hence, the QoI request can be satisfied. 

