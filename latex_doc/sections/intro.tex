
\section{Introduction}
\label{sec:intro}

%area
Symptotic analysis is an emerging field of characterizing practical network scalability instead of the common asymptotic analysis.  Introduced in \cite{scalability_manets_theory_vs_practice}, symptotics is extremely useful for applications and network designers that are interested in determining the limits of a specific network implementation as well as how various factors affect these limits in terms of scalability.  For example, if one is designing an emergency mesh network to quickly install to replace destroyed infrastructure after a natural disaster, understanding the traffic limitations or what topology allows for the largest feasible network size is crucial to successful design.

%problem
%why not solved
While including the consideration of Quality of Information (QoI) into network control protocols or analysis is not novel, its limitations and effects on network scalability have not been considered before.  This consideration is extremely important, though, because in many recent fields of study, QoI is being used as the metric of a network's capabilities.  Increasingly, network nodes are becoming more capable of affecting QoI through data fusion, compression, selection, etc.  This work aims to begin understanding the connections between these actions and practical network scalability.

%insight
Specifically, in this work we consider the practical effects of real networks as protocol overhead, contention, rates in conjunction with QoI.  While adopting a general framework for QoI, we use timely, similarity-based image collection to provide motivation and concrete applications with results.  

%contribution
Our results first provide maximum sizes to which the networks can scale, given requested levels of QoI. The QoI depends on timeliness and one of two metrics considered: completeness or diversity.  These attributes are defined by the sum similarity of collected images resulting from Top-K queries for completeness, and sum dissimilarity of images collected using the greedy spanner algorithm. These definitions will be explained in more detail in Section \ref{sec:qoi_model}.  We observe that the network scalability considerably reduces with higher completeness and diversity requests, as well as with stringent timeliness requirements.  

Finally, we identify the trade-off of a given QoI requirement resulting in both a minimum required network size to provide a required number of images and the maximum feasible network size able to support that amount of necessary traffic.  We identify the region of QoI requests where the former does not exceed the latter, and, hence, the QoI request can be satisfied. 

