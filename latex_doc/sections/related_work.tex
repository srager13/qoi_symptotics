
\section{Related Work}
\label{sec:related_work}

%we could start by saying photonet considers similartiy-based image collection(diversity metric), and mediascope considers different queries and timeliness, but the transmission rates were not detailed. here, we consider an actual network etc.

We adopt the symptotic scalability framework \cite{scalability_manets_theory_vs_practice}, which has been previously applied to content-agnostic static networks \cite{symptotics_framework_scalability} and mobile networks \cite{scal_analysis_mobility}.  Other works that characterize capacity of wireless networks, like \cite{li_capacity, gupta2000capacity, nom_cap_wmns}, do so different by considering how networks scale asymptotically or analyzing only one specific network setup instead of developing a general model for scalability.

A large number of works provide definitions for and frameworks that utilize Quality of Information.  We will address only the most relevant ones here.  \cite{qoi_aware_tactical_mil_nets} \cite{qoi_aware_trx_pol_time_vary_links} specify a framework called Operational Information Content Capacity, which describes a framework of the obtainable region of QoI, a notion similar to the \emph{scalably feasible QoI region} in Section \ref{sec:qoi_scalability}.  These approaches use a more general network model, though, and do not provide any method for determining the possible size of the network or impact of various network design choices like medium access protocols.  

QoI-based scheduling has been considered from a number of various angles, including control choices of data selection \cite{dcoss_max_cov, opt_qoi_data_collection_bijarbooneh}, routing \cite{quality_aware_routing_tan}, and scheduling/rate control \cite{qoi_sched_task_proc_nets, toward_qoi_rate_control}.  It is also the focus of a credibility-aware optimization technique in \cite{social_swarming}.  Work in \cite{qoi_aware_mobile_apps} is related in that it evaluates the impact of varying QoI requirements on usage of network resources.  These works all provide insights into QoI that we use, but our focus is on understanding the impact of QoI requirements on network size, not on designing network protocols as these works do.

Similarity-based image collection has previously been considered \cite{photonet} and \cite{mediascope}. In \cite{photonet}, authors consider a DTN network where the objective is to collect the most diverse set of pictures at every node.  Authors consider a picture prioritization and dropping mechanism in order to maximize the diversity, defined by dissimilarities of the collection of pictures. However, it does not consider attributes of timeliness, nor the consideration of transmission rates and network topology.  \cite{mediascope} considers a smartphone application where different queries called top-K, spanner, and K-means clustering are defined.  Each of these queries are based on image similarity metrics, and we use top-k and spanner here. While timeliness is considered as an objective in this work, the effects of rates and network topologies are overlooked.

NOTE:  Need more on these???:  Time varying queues, QoI outage, Freshness-based for multiple sensors.  
