
\section{Related Work}
\label{sec:related_work}

%we could start by saying photonet considers similartiy-based image collection(diversity metric), and mediascope considers different queries and timeliness, but the transmission rates were not detailed. here, we consider an actual network etc.

We adopt the symptotic scalability framework from \cite{scalability_manets_theory_vs_practice}, which has been previously applied to content-agnostic static networks \cite{symptotics_framework_scalability} and mobile networks \cite{scal_analysis_mobility}.  Other works that characterize the capacity of wireless networks, like \cite{li_capacity, gupta2000capacity, nom_cap_wmns}, do so differently by considering how networks scale asymptotically or by analyzing specific network instances instead of developing a general model for scalability.

A large number of works provide definitions for Quality of Information and frameworks utilizing it.  We will address only the most relevant ones here.  Primarily, QoI has been used in scheduling and has been considered from a number of various angles, including control choices of data selection \cite{dcoss_max_cov, opt_qoi_data_collection_bijarbooneh}, routing \cite{quality_aware_routing_tan}, and scheduling/rate control \cite{qoi_aware_trx_pol_time_vary_links, toward_qoi_rate_control,explor_vs_exploit, qoi_outage}.  It is also the focus of a credibility-aware optimization technique in \cite{social_swarming}.  

The work in \cite{qoi_aware_mobile_apps} evaluates the impact of varying QoI requirements on usage of network resources, which is certainly related to this paper.  Our focus is on a broader scale than this work, though, by modeling an entire network instead of a single node as the authors in \cite{qoi_aware_mobile_apps}.

Additionally, \cite{qoi_aware_tactical_mil_nets} and \cite{oiccm} outline a framework called Operational Information Content Capacity, which describes the obtainable region of QoI, a notion similar to the \emph{scalably feasible QoI region} in Section \ref{sec:qoi_scalability}.  These approaches use a general network model, though, and do not provide any method for determining the possible size of the network or impact of various network design choices like medium access protocols.   % might need to look at these two papers again

Similarity-based image collection has previously been considered \cite{photonet} and \cite{mediascope}. In \cite{photonet}, authors consider a DTN network where the objective is to collect the most diverse set of pictures at every node.  Authors consider a picture prioritization and dropping mechanism in order to maximize the diversity, defined by dissimilarities of the collection of pictures. However, it does not consider attributes of timeliness, nor the consideration of transmission rates and network topology.  \cite{mediascope} considers a smartphone application where different queries called top-K, spanner, and K-means clustering are defined.  Each of these queries are based on image similarity metrics, and our use of Top-K and Spanner algorithms here was inspired by this paper.  While timeliness is considered as an objective in this work, the effects of rates and network topologies are overlooked.

%NOTE:  Need more on these???:  Time varying queues, QoI outage, Freshness-based for multiple sensors.  
